\documentclass[11pt, a4paper]{article}
\usepackage[left=1.4cm,text={18.2cm, 25.2cm},top=2.3cm]{geometry}

\usepackage{times}
\usepackage[czech]{babel}
\usepackage[IL2]{fontenc}
\usepackage[utf8]{inputenc}
\usepackage{listings}

\usepackage{amsmath, amsthm, amssymb}
\usepackage[bottom]{footmisc}
\usepackage{graphicx}
\usepackage{hyperref}
\usepackage{geometry}
\geometry{a4paper}

\begin{document}

\begin{titlepage}
\thispagestyle{empty}

    \begin{center}

        {\Huge \textsc{Vysoké učení technické v~Brně \\[0.5em]}}

        {\huge \textsc{Fakulta informačních technologií}}

        \vspace{\stretch{0.382}}

        {\Large Mikroprocesorové a vstavané systémy \\[0.5em]
        \LARGE Digitálne FM rádio (modul RDA5807M a zosilovač PAM8407) 
         }

        \vspace{\stretch{0.618}}

    \end{center}
{\Large 2024 \hfill Boris Hatala (xhatal02)}

\end{titlepage}

\newpage
\tableofcontents
\newpage


\section{Úvod}
Táto dokumentácia opisuje návrh a implementáciu digitálneho FM rádia postaveného 
na hardvéri ESP32 a ďalších súčastiach.

\subsection{Cieľ projektu}
Cieľom je vytvoriť funkčné digitálne FM rádio, 
ktoré umožňuje používateľovi nastavovať frekvenciu v pásme 76MHz až 108Mhz FM a 
ovládať hlasitosť.

\section{Použité metódy a technológie}
\subsection{Hardvér}
\begin{itemize}
    \item Doska ESP32 Wemos D1 R32\footnote{\url{https://docs.platformio.org/en/latest/boards/espressif32/wemos_d1_uno32.html}}
    \item Modul s OLED displejom 0,96`` \-\ Adafruit\footnote{\url{https://www.hadex.cz/m508a-displej-oled-096-128x64-znaku-7pinu-bily/}}
    \item FM príjmač na bázi čipsetu RDA5807M\footnote{\url{https://www.hadex.cz/m501a-fm-prijimac-pro-arduino-modul-rrd102-v20-io-rda5807m/}}
    \item Audio zosilovač PAM8407D\footnote{\url{https://www.diodes.com/assets/Datasheets/products_inactive_data/PAM8407.pdf}}
    \item Rotačný enkodér KY-040\footnote{\url{https://elty.pl/cs_CZ/p/Impulsni-modul-snimace-KY-040/1155}}
\end{itemize}
\subsection{Software}
Program je napísaný v jazyku XXX vo vývojovom prostredí XXX. Boli použité knižnice XXX\dots

\section{Použitie}

\section{Implementačné detaily}
\subsection{Zapojenie}
\subsubsection{Pripojenie FM modulu RDA5807M}

\begin{tabular}{|c|c|}
\hline
\textbf{FM Pin} & \textbf{ESP32 Pin} \\
\hline
VDD & 3.3V \\
GND & GND \\
SDA & GPIO21 (SDA) \\
SCK & GPIO22 (SCL) \\
RCH & RINP (Pin 01 na PAM8407) \\
LCH & LINP (Pin 08 na PAM8407) \\
ANT & Drôtová anténa (cca 10–15 cm) \\
\hline
\end{tabular}

\subsubsection{Pripojenie OLED displeja (SPI rozhranie)}

\begin{tabular}{|c|c|}
\hline
\textbf{OLED Pin} & \textbf{ESP32 Pin} \\
\hline
GND & GND \\
VCC & 3.3V alebo 5V (podľa OLED displeja) \\
D0 (Clock) & GPIO18 (SCK) \\
D1 (Data: SDA or MOSI)& GPIO23 (MOSI) \\
RES & GPIO16 \\
DC & GPIO17 \\
CS & GPIO5 \\
\hline
\end{tabular}
\\

\subsubsection{Pripojenie zosilňovača PAM8407}

\textbf{Vstupy (z FM modulu):}

\begin{tabular}{|c|c|c|}
\hline
\textbf{PAM Pin}    & \textbf{FM Pin} & \textbf{ESP32 Pin} \\
\hline
RINP (Pin 01)       & RCH                          & -- \\
LINP (Pin 08)       & LCH                          & -- \\
RINN (Pin 02)       & --                          & GND \\
LINN (Pin 07)       & --                          & GND \\
VDD                 & --                            & 5V \\
GND                 & --                            & GND \\
\hline
\end{tabular}
    

\subsubsection{Pripojenie enkodéru KY-040}

\begin{tabular}{|c|c|}
\hline
\textbf{Enkodér Pin} & \textbf{ESP32 Pin} \\
\hline
CLK & GPIO33 \\
DT  & GPIO32 \\
SW  & GPIO25 \\
+   & 3.3V \\
GND & GND \\
\hline
\end{tabular}

\section{Záver}



\end{document}